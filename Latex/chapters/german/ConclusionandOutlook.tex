
Ziel dieser Arbeit war es Das Aufstiegsverhalten in Kapillaren zu untersuchen, die nur wenige Nanomenter im durchmesser sind. Dazu wurden mehrere Simulationen durchgeführt, die unterschiedliche Kontatkwinkel und dessen Einfluss auf das Wachstum untersuchen, oder auch die Auswirkung eines relaxationsfaktors 



Dazu wurden zunächst in Kapitel \ref{chap: wettingTheory} die Grundlagen der Benetzung und des Kapillaren aufstiegs, sowie derzeitige Kenntnisse vorgestellt. Anschließend wurde in Kapitel \ref*{chap: PhaseFieldMethod} ein Überblick über die Phasenfeld Methode gegeben und in Kapitel \ref*{chap: CaseSetup} die Umsetzung mit dem verwendeten Solver \texttt{phaseFieldFoam} beschrieben. 


Ziel dieser Arbeit war es den aufstieg einer Wassersäule in Nonometer großen Kapillaren zu untersuchen. Dies wurde soweit bekannt bisher noch nciht mit einer Phasenfeld Methode durchgeführt. Dazu wurde der solver \texttt{phaseFieldFoam} verwendet und auf dem Lichtenberg High performance computer der TU Darmstadt ausgeführt. Die untersuchung des Aufstiegs fokussierte sich dabei auf das Wachstumsverhalten, genauer auf den Übergang des linearen Wachstums hin zum diffusiven wachstum, welches von Lucas und Washburn beschrieben wurde. Dazu wurden zunächst in Kapitel \ref*{chap: wettingTheory} die notwendigen Grundlagen der Benetzung und des Kapillaren aufstiegs vorgestellt. Da der Kapillare aufstieg für viele Anwendungen von großem interesse ist, ist dieser auch gegenstand vieler Untersuchungen. Einige dieser Untersuchungen wurden vorgestellt und bildeten die die untersuchungsgrundlage dieser Arbeit. Anschließend wurde in Kapitel \ref*{chap: PhaseFieldMethod} ein überblick über die verwendete Phasenfeld methode im geiste von Cahn und Hillard gegeben. Die verwendete Geometrie der Kapillare und die notwendigen Randbedingungen und Parameter einestellungen zur Simulation und dessen Auswertung wurden in Kapitel \ref*{chap: CaseSetup} beschrieben. \todo{validation}
Wie abschließend in Kapitel \ref*{chap: Results} gezeigt wurden zur Untersuchung des Aufstiegsverhaltens mehrere Simulationen mit unterschiedlichen Randbedingunen durchgeführt, um mögliche einflussfaktoren ermitteln zu können. 


\begin{itemize}
    \item complex geometry
    \item temperature dependent capillary rise?
    \item surfactants and capillary rise?
    \item more fluids with other viscosisties
\end{itemize}