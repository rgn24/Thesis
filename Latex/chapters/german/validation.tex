The \texttt{phaseFieldFoam}- Solver 

Da solche Simulationen für so kleine Geometrien jedoch noch soweit bekannt bisher noch nicht mit einer Phasenfeld Methode durchgeführt wurden, 

\begin{itemize}
    \item Moradi2021 Laplace Test
\end{itemize}


Der Solver \texttt{phaseFieldFoam} wurde bereits mehrfach für unterschiedliche Probleme validiert. Auch Simulationen mit einer Wedge wurden in \cite{holzinger2021DirectNumericalSimulation} mit Adaptiver Netzverfeinerung durchgeführt. Simulationen von Kapillaren oder auch parallelen Platten wurden in Hagg \cite{hagg2019DirekteNumerischeSimulation}, bzw. von Cai et al. \cite{cai2015NumericalSimulationWetting} durchgeführt. Samkhaniani et al. \cite{samkhaniani2021BouncingDropImpingement} simulierte einen springenden Tropfen auf einer hydrophoben Oberfläche. Viele weitere Interessante und wichtige Simulationen und Validierungen wurden mit \texttt{phaseFieldFoam} durchgeführt \cite{bodziony2023StressfulWayDroplets,yinDirectNumericalSimulation,worner2021SpreadingReboundDynamics,bagheriInterfacialRelaxationCrucial2022}, dessen einzelne Nennung den Rahmen dieser Arbeit sprengen würde. 
Da der solver bereits an vielen Stellen validiert wurde, wird in dieser Arbeit lediglich ein Laplace-Test durchgeführt. Dabei kann mit Gleichung \ref{eq: YoungLaplaceEQ} die Druckdifferenz bei einem sich einstellenden Radius des Interfaces der Simulation berechnet und mit dem Theoretischen wert Verglichen werden. Für diesen Vergleich muss der Radius des Interfaces in der Simulation bekannt sein. Um diesen zu erhalten, kann unter der Annahme, dass der Mittelpunkt des Kreises auf der Rotationsachse liegt, der Radius mit 









