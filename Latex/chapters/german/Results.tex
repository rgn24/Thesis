\todo[inline]{The Position of the meniskus was exported with paraview in the decomposed state. The position was then extracted with a python script.Probe data was preprocessed as well and several computations for eval of simulations. Plots were generated with matplotlib?; Start of real results with an overview when what where. Maybe work with normalized data? }


Wie in Kapitel \ref{sec: capillaryRise} gezeigt, ist der Kapillare Aufstieg bis heute nicht verstanden. Ziel dieser Arbeit ist es daher 


Wie bereits beschrieben soll der Übergang vom linearen Anstieg der höhe einer Wassersäule in einer Kapillare hin zum Lucas Washburn Regime Untersucht werden. Weiter wurde beschrieben wie Delanoy et al. \cite{delannoy2019DualRoleViscosity} oder Ruiz et al. \cite{ruiz-gutierrez2022LongCrossoverDynamics} das Wachstum beschrieben. Zum Vergleich damit werden zunächst die Zeitpunkte oder Längen anhand ihrer vorhersagen berechnet und anschließend verglichen. Zunächst aber ein direkter Vergleich mit der Lucas-Washburn Gleichung. 