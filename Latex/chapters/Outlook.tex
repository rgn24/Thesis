\begin{itemize}
    \item complex geometry
    \item temperature dependent capillary rise?
    \item surfactants and capillary rise?
    \item more fluids with other viscosisties
    \begin{itemize}
        \item as delanoy discussed the viscosity plays a major role in the capillary rise
    \end{itemize}
    \item in depth reseach of gamma value and possible dependence of mesh resolution
\end{itemize}



\section{Complex Geometry}
In experiments it is not possible to generate a capillary with a constant diameter. Therefore a geometry with a more realistic geometry is also simulated. To create the porous medium for the experiments, "spheres" are used in a pour. The area between the spheres of this fill is then filled and the spheres are dissolved. This creates capillaries that are not ideally smooth as previously assumed, but rather a shape like in figure (\todo[inline]{ref to a picture with used geometry}). It was assumed in a simplified way that no bridges are created between capillaries in this process. Furthermore, it is assumed that the spheres are ideally aligned with each other at the beginning of the process.  \todo{Check, if correct!} 
In this case it is assumed, that the centre of the circle is in the middle of a segment and only displaced by an offset from the x-axis. The connection between the segments is to make sure the mesh is sufficient. 

The numerical setting keep the same as in the wedge case with a cylindrical capillary. 


To simulate the imbibition process we use the \verb|OpenFoam| framework. To get a estimation of the necessary parameters, a previously validated case is reused and adapted to this case. First the simulation is assumed to be axissymmetric. Therefore a Wedge is used for the major part of the simulations. To validate the results of the wedge a 3D simulation was performed, too. 

The necessary geometry, settings and boundary conditions for each case are described below. 

\section{Temperature depend capillary rise}
Bisher wurden nur isotherme simulationen durchgeführt. Die abhängigkeit des Kapillaren aufstiegs mit berücksichtigung der Temperatur wäre aber daher sinnvoll, da die meisten Technischen anwendungen nicht isotherm arbeiten. Da die Arbeit von Delanoy et al. \cite{delannoy2019DualRoleViscosity} darauf hinweist, dass die Viskosität des systems einen großen einfluss auf den imbibitionsprozess haben kann und die viskosität bekanntermaßen stark temperaturabhängig ist, wäre die abhängigkeit von dieser wichtig, um mögliche effekte durch abkühlung oder erwärmung des Fluids zu erkennen und ggf. sogar zu nutze zu machen, um Oberflächeneffekte auszunutzen beim transport des Fluids, um auf pumpen idealerweise verzichten zu können. 
Daher wäre zu beginn eine genauere Untersuchtung unterschiedlicher Viskositäten auf den Kapillaren aufstieg möglicherweise ein guter einstieg für eine Untersuchung der Tempetaturbedingten veränderungen. 


\section{surfactants influence on capillary rise}
Auch der einfluss von tensiden auf den Kapillaren aufstieg kann von großen interesse sein, da auch Reaktionen an der Oberfläche gegebenenfalls ausgenutzt werden können.

\section{Phase Field Parameters}
Um eine Simulation mit einer Phasenfeld Simulation durchführen zu können, müssen je nach experiment mehrere Parameter definiert werden. In dieser Arbeit musste die Dicke des Interfaces, die mobilität und Wand relaxation vorgegeben werden. Parameter für mobilität und Wand relaxation zu finden ist bisher mit viel aufwand und vielen Simulaitonen verbunden, die im schlimmsten Fall die vorteile durch schnellere Simulationen im Vergleich zu Molecular Dynamik Simulaitonen für so kleine Probleme nicht mehr überwiegen können. Daher wäre das Ziel eine korrelation der Parameter zu finden eine der größten vereinfachungen für simulationemn und würde viele Prozesse deutlich beschleunigen. Weiter wären auch weitere Untersuchungen des Wand relaxationsparameters gepaart mit experimenten und bekannten Oberflächenbeschaffenheiten interessant. Das Dabei viele Faktoren berücksichtigt werden müssen, wie mögliche interatktionen der Materialien oder Temperaturen ist wichtig zu betrachten und Berücksichtigen. Dennoch würde siene solche untersuchung die aufwendige arbeit erleichtern korrekte Parameter zu finden, die die Simulaiton benötigt. 