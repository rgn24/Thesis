It is assumed that basic knowledge of wetting, capillary rise, and the phase field method is available. Otherwise, the reader is referred to the respective chapters.

\section{Rising Dynamics}

With the assumptions introduced in Chapter \ref{chap: Introduction}, the following investigations will always be based on them. The dynamics of rise in a capillary have been of great interest for a long time. Bell and Cameron already empirically demonstrated in 1906 \cite{bell1906FlowLiquidsCapillary} that the rise in a capillary can be described with \(z(t)^n = Kt\), with \(n\) and \(K\) as temperature-dependent constants and \(z\) as the current height of the meniscus in the capillary. Lucas \cite{lucas1918UeberZeitgesetzKapillaren} in 1918 and Washburn \cite{washburn1921DynamicsCapillaryFlow} in 1921 independently derived an equation that used material properties or measurable quantities to describe the rise of the capillary. However, effects due to gravity or inertia were neglected in their derivation. Thus, they developed a \(z(t) \sim \sqrt{t}\) relationship, which is often referred to as Lucas-Washburn behavior or Lucas-Washburn dynamics. In this context, the contact angle is an important quantity, which they assumed to be an equilibrium contact angle.
Bosanquet \cite{bosanquet1923LVFlowLiquids} pointed out in his work that this equation will lead to unphysical results as \( t \) approaches zero.
Siegel \cite{siegel1961TransientCapillaryRise} was the first to demonstrate a linear relationship with time in an experiment under microgravity conditions, although he did not reach the Lucas-Washburn regime. Zhmud et al. \cite{zhmud2000DynamicsCapillaryRise} described a quadratic relationship for the region where the fluid is drawn into the capillary, followed by the \( \sqrt{t} \) behavior predicted by Lucas and Washburn.
Dreyer et al. \cite{dreyer1994CapillaryRiseLiquid} examined parallel plates under microgravity conditions and divided the rise of the meniscus into three regions: starting with a quadratic region, followed by a linear region, and finally the \( \sqrt{t} \) region.
Quéré \cite{quere1997InertialCapillarity} demonstrated an \( h \sim t \) behavior at the beginning. Stange \cite{stange2003CapillaryDrivenFlow} confirmed the three regions already described by Dreyer et al. \cite{dreyer1994CapillaryRiseLiquid} and derived equations using dimensionless numbers and dynamic contact angles to develop transition points.
Fries et al. \cite{fries2008TransitionInertialViscous} categorized the rise into areas where different forces are at play. Initially, inertia dominates, followed by a transition region where viscous forces come into play, until eventually, the viscous forces prevail. They also derived dimensionless time points at which these transitions occur.
Dellanoy et al. \cite{delannoy2019DualRoleViscosity} built upon this by conducting experiments that showed a linear initial regime. They also addressed the issue, previously noted by Washburn\cite{washburn1921DynamicsCapillaryFlow}, that the capillary must be pre-wetted for the results to align with the Lucas-Washburn equation. Additionally, the influence of fluid viscosity on the rising behavior was observed and explained by dissipative regions near the contact line.
Ultimately, Ruiz-Gutiérrez et al. \cite{ruiz-gutierrez2022LongCrossoverDynamics} investigated the rise dynamics for various materials and identified a material dependency. Each of the materials used reached the Lucas-Washburn regime, but only some went through all the mentioned regimes on their way there. \todo{Elaborate further?}



\section{Contact Line Problem}
\label{sec: ContactLineProblem}

The contact line is the line formed at the intersection of three phases. In the case of a capillary, it occurs at the transition between the two fluids and their meeting point on the wall. As Blake notes in his work \cite{blake2006PhysicsMovingWetting}, the problem of dynamic contact lines involves both macroscopic and microscopic dimensions and is usually described by quantities such as the velocity of the contact line, viscosity, surface tension, and the contact angle of the fluid with the wall \cite{blake2006PhysicsMovingWetting, voinovHydrodynamicsWetting1977, cox1986DynamicsSpreadingLiquids}\todo{cite possibly Karim2022?}.


There are several approaches to describing the dynamic contact angle, each based on different methods. In the following, only essential aspects will be briefly discussed that are necessary to demonstrate how the phase-field method is suitable and how it differs from other methods. For more detailed descriptions of the Molecular Kinetic Method and the Hydrodynamic Method, the reader is referred to the work by Blake \cite{blake2006PhysicsMovingWetting} and a more recent review that includes additional methods, as well as the phase-field method, by Karim \cite{mohammadkarim2022ReviewPhysicsMoving}. Since these methods are often closely related to subsequent experiments or simulations, the following will also directly address their relevance to simulations.


Due to the singularity of the Navier-Stokes equation at the moving contact line, Yue et al. \cite{yue2010SharpinterfaceLimitCahn} demonstrated in their work that a diffusive interface is capable of regulating this singularity. They also showed a close connection to the hydrodynamic theory of Cox \cite{cox1986DynamicsSpreadingLiquids}.

\section{Phase Field Method Simulations}

In 1999, Jacqmin \cite{jacqmin1999CalculationTwoPhaseNavier} extended the Navier-Stokes equations with \(-C\nabla \phi\) in his work and replaced the interface advection equation with an advective-diffusive function. Here, \(C\) is an order parameter indicating which phase is present, and \(\phi\) is the chemical potential of \(C\). He was able to demonstrate that simulations using this method can perform precise calculations where the interface can be smaller than two cell widths. Subsequently, Jacqmin showed in the following work \cite{jacqmin2000ContactlineDynamicsDiffuse} that the potential of this approach can also be applied to contact line problems. The singularities of the hydrodynamic model at the contact line shown by \todo{cite huh scr... somehow} using adhesion conditions do not occur. Moreover, the results of the phase field method approach those of the sharp interface methods.